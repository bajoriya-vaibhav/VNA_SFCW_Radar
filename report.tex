\documentclass[11pt,a4paper]{article}
\usepackage{float}
\usepackage{tikz}
\usetikzlibrary{shapes, arrows, positioning}
\usepackage[utf8x]{inputenc}
\usepackage[T1]{fontenc}
\usepackage{amsmath}
\usepackage{amssymb}
\usepackage{booktabs}
\usepackage{mathptmx} % Use Times Font
\usepackage{graphicx}
\usepackage[colorlinks=true,linkcolor=blue,citecolor=blue]{hyperref}
\usepackage{calc}
\usepackage{enumitem}
\usepackage{multirow}
\usepackage{subfig}
\usepackage{siunitx}

\frenchspacing
\linespread{1.2}
\usepackage[a4paper, lmargin=0.1666\paperwidth, rmargin=0.1666\paperwidth, tmargin=0.1111\paperheight, bmargin=0.1111\paperheight]{geometry}

\usepackage[all]{nowidow}
\usepackage[protrusion=true,expansion=true]{microtype}

\hypersetup{
pdfsubject = {Target Detection using SFCW Radar},
pdftitle = {Analysis of Target Detection Using SFCW Radar with Vector Network Analyzer},
pdfauthor = {Student Name}
}

\begin{document}

\begin{titlepage}
\newcommand{\HRule}{\rule{\linewidth}{0.5mm}}

\center 
 
\textsc{\LARGE University Name}\\[1cm]

\textsc{\Large Advanced Radar Systems}\\[0.2cm]
\textsc{\large ECE 5XX}\\[1cm] 									

\HRule \\[0.8cm]
{ \huge \bfseries \huge{Target Detection Using Stepped-Frequency\\Continuous-Wave Radar with Vector\\Network Analyzer}}\\[0.7cm]						

\HRule \\[2cm]

\large
\emph{Student Name}
 (ID: XXXXXXXX)\\[1cm]


{\large May 2025}\\[3cm]
% \includegraphics[width=0.22\textwidth]{university_logo.png}\\[0.3cm] 	
\vfill 
\end{titlepage}

\tableofcontents
\newpage

\section{Introduction}
Stepped-Frequency Continuous-Wave (SFCW) radar technology has emerged as a powerful tool for subsurface target detection. Unlike conventional pulsed radar systems that transmit short, high-power pulses, or wideband Frequency Modulated Continuous Wave (FMCW) radars with continuously varying frequency, SFCW radar operates through a fundamentally different principle. This report presents the design, implementation, and testing of a target detection system using a Vector Network Analyzer (VNA) based on SFCW principles.

In an SFCW system, the radar sequentially transmits a series of discrete, narrowband continuous wave signals across carefully selected frequency steps. Through this approach, SFCW radar synthesizes an effective wideband operation that rivals broadband systems, while maintaining significant advantages in hardware implementation, cost-effectiveness, and signal processing flexibility. This makes SFCW particularly well-suited for applications requiring high range resolution and the ability to detect targets in complex environments.

The core concept of SFCW radar rests on the coherent integration of phase information acquired across multiple discrete frequencies, effectively creating a synthetic bandwidth that determines the system's range resolution. This frequency-domain approach offers remarkable flexibility in adapting operational parameters to specific subsurface environments—a critical advantage when dealing with varying soil conditions, moisture content, and target characteristics.

This report details the use of a PocketVNA device to implement an SFCW radar system for target detection. We explore the theoretical foundations, calibration procedures, signal processing techniques, and experimental results of our implementation.

\section{Theory of Stepped-Frequency Continuous-Wave (SFCW) Radar}

\subsection{Fundamental Principles and Signal Model}

Stepped-Frequency Continuous-Wave (SFCW) radar operates by transmitting a sequence of $N$ discrete frequencies, often referred to as "frequency bursts," spanning a total bandwidth $B$. Each frequency step represents a single narrowband continuous wave (CW) transmission, and collectively these steps synthesize a wideband operation.

Each frequency in an SFCW radar follows a linear progression given by:
\begin{equation}
    f_n = f_0 + n \Delta f,\quad n = 0, 1, 2, \ldots, N-1
\end{equation}

where:
\begin{itemize}
    \item $f_0$ represents the initial (lowest) frequency
    \item $N$ is the total number of frequency steps
    \item $\Delta f$ is the frequency step size
\end{itemize}

The step frequency is calculated using:
\begin{equation}
    \Delta f = \frac{B}{N-1}
\end{equation}

Each frequency is transmitted for a predetermined dwell time $T_d$, during which the radar receiver measures both amplitude and phase of the returned signal.

\subsection{Signal Propagation and Target Interaction}

When an SFCW radar signal propagates through a medium (air, soil, concrete, etc.) and encounters a target, the signal undergoes various interactions, including:
\begin{itemize}
    \item \textbf{Transmission losses} as the signal penetrates any medium interfaces
    \item \textbf{Multiple reflections} between interfaces and targets
    \item \textbf{Attenuation} based on the material properties
\end{itemize}

For GPR applications specifically, the signal propagation is complicated by the varying dielectric properties of soil or concrete, which depend strongly on moisture content, mineralogical composition, and density. The propagation velocity $v$ in the medium relates to the dielectric constant $\varepsilon_r$ by:

\begin{equation}
    v = \frac{c}{\sqrt{\varepsilon_r}}
\end{equation}

where $c$ is the speed of light in vacuum. This velocity variation must be accounted for in accurate depth estimation.

\subsection{Signal Reception and Phase Processing}

The phase information contained in received signals provides critical information regarding the range to the target. For a target at range $R$, the round-trip time delay $\tau$ is:

\begin{equation}
    \tau = \frac{2R}{v} = \frac{2R\sqrt{\varepsilon_r}}{c}
\end{equation}

This time delay manifests as a frequency-dependent phase shift in the received signal at each frequency step:

\begin{equation}
    \phi_n = 2\pi f_n \tau = 2\pi \left( f_0 + n\Delta f \right) \frac{2R\sqrt{\varepsilon_r}}{c}
\end{equation}

We can decompose this phase shift into two components:
\begin{align}
    \phi_n &= \frac{4\pi f_0 R\sqrt{\varepsilon_r}}{c} + \frac{4\pi n\Delta f R\sqrt{\varepsilon_r}}{c}\\
    &= \phi_0 + n \cdot \Delta\phi
\end{align}

The first term represents a constant phase offset, while the second term creates a linear phase progression across frequency steps that is proportional to the target range. This linear phase variation is the key to extracting range information through frequency-domain processing.

For a point target with radar cross section $\sigma$ at range $R$, the complex received signal at each frequency step can be modeled as:

\begin{equation}
    S_n = A_0 \sigma \frac{1}{R^4} e^{-j\phi_n} + \eta_n
\end{equation}

where $A_0$ incorporates transmit power, antenna gains and system constants, the $R^4$ term accounts for two-way propagation loss, and $\eta_n$ represents noise.

\subsection{Range Profile Extraction}

By analyzing the phase shifts across all frequency steps, the radar constructs a range profile. An Inverse Fast Fourier Transform (IFFT) converts the frequency-domain responses into a time-domain (range-domain) representation, yielding high-resolution target localization.
\begin{equation}
    s(t) = \text{IFFT}\left\{ S(f_n) e^{j\phi_n} \right\}
\end{equation}

Windowing techniques (like Hanning, Kaiser) can be applied to reduce sidelobes and improve resolution, though at the expense of slightly degraded range resolution.

\subsection{Key Performance Parameters}

The performance of an SFCW radar system can be characterized through several key metrics:

\begin{itemize}
    \item \textbf{Range Resolution} ($\Delta R$): The system's ability to distinguish between targets at similar ranges:
    \begin{equation}
        \Delta R = \frac{c}{2B\sqrt{\varepsilon_r}}
    \end{equation}
    
    \item \textbf{Maximum Unambiguous Range} ($R_{max}$): The maximum range that can be measured without ambiguity:
    \begin{equation}
        R_{max} = \frac{c}{2\Delta f \sqrt{\varepsilon_r}}
    \end{equation}
    
    \item \textbf{Processing Gain} ($G_p$): The improvement in signal-to-noise ratio achieved through coherent integration:
    \begin{equation}
        G_p = 10\log_{10}(N) \text{ dB}
    \end{equation}
    
    \item \textbf{Penetration Depth}: Highly dependent on medium properties, frequency range, and transmit power, typically ranging from tens of centimeters to several meters for GPR applications.
\end{itemize}

\section{Vector Network Analyzer Fundamentals}

\subsection{S-Parameters}

S-parameters (scattering parameters) are used to characterize high-frequency networks and components. For our target detection system, the most relevant S-parameters are:

\begin{itemize}
    \item \textbf{S11}: The reflection coefficient measured at port 1, representing the ratio between the incident and reflected wave. For our target detection application, S11 measurements indicate reflections from targets.
    
    \item \textbf{S21}: The forward transmission coefficient from port 1 to port 2, representing how much of the signal transmitted from port 1 is received at port 2. This parameter is relevant for through-transmission measurements.
\end{itemize}

The relationship between S-parameters and physical quantities can be expressed as:
\begin{align}
    S_{11} &= \frac{b_1}{a_1}|_{a_2=0} \\
    S_{21} &= \frac{b_2}{a_1}|_{a_2=0}
\end{align}

where $a_i$ represents the incident wave at port $i$ and $b_i$ represents the reflected/transmitted wave at port $i$.

Other useful parameters derived from S-parameters include:
\begin{align}
    \text{Impedance: } Z &= Z_0 \cdot \frac{1+S_{11}}{1-S_{11}} \text{ (for a one-port measurement)} \\
    \text{VSWR} &= \frac{1+|S_{11}|}{1-|S_{11}|} \\
    \text{Return Loss (dB)} &= -20\log_{10}|S_{11}| \\
    \text{Insertion Loss (dB)} &= -20\log_{10}|S_{21}|
\end{align}

where $Z_0$ is the characteristic impedance, typically 50 ohms.

\section{Experimental Setup}

\subsection{Hardware Components}

Our target detection system consists of the following components:

\begin{itemize}
    \item PocketVNA: A portable 2-port vector network analyzer with frequency range 500 kHz - 4 GHz
    \item Two wideband antennas (horn or Vivaldi type) with operational frequency matching our selected band
    \item Coaxial cables with SMA connectors to connect the antennas to the VNA
    \item Calibration kit consisting of Open, Short, Load, and Through standards
    \item Test targets of various sizes and shapes
    \item Computer with PocketVNA software for control and data acquisition
\end{itemize}

\subsection{Physical Layout}

The experimental setup follows a monostatic radar configuration where a single antenna is used for both transmission and reception. The antenna is connected to port 1 of the PocketVNA, and the following configurations were tested:

\begin{itemize}
    \item \textbf{Configuration 1 (Reflection mode)}: Single antenna connected to port 1, measuring S11
    \item \textbf{Configuration 2 (Transmission mode)}: Two antennas, one connected to port 1 and another to port 2, measuring S21
\end{itemize}

Targets were placed at various ranges from the antenna(s) to verify detection capability and range resolution.

\subsection{SFCW Parameter Selection}

For our target detection experiment, we selected the following SFCW parameters:

\begin{itemize}
    \item Start frequency ($f_0$): 1600 MHz
    \item Stop frequency ($f_N$): 2500 MHz
    \item Number of steps ($N$): 1001
    \item Step size ($\Delta f$): 0.9 MHz
    \item Synthetic bandwidth ($B$): 900 MHz
\end{itemize}

These parameters provide a theoretical range resolution of approximately:
\begin{equation}
    \Delta R = \frac{c}{2B} = \frac{3 \times 10^8}{2 \times 9 \times 10^8} = 0.167 \text{ meters} = 16.7 \text{ cm}
\end{equation}

The maximum unambiguous range is:
\begin{equation}
    R_{max} = \frac{c}{2\Delta f} = \frac{3 \times 10^8}{2 \times 0.9 \times 10^6} = 166.7 \text{ meters}
\end{equation}

\section{Calibration Procedure}

Proper calibration is essential for accurate measurements using a VNA. The calibration process compensates for systematic errors in the measurement setup, including:

\begin{itemize}
    \item Cable losses and phase shifts
    \item Connector mismatches
    \item Internal reflections in the VNA
    \item Cross-talk between ports
\end{itemize}

\subsection{Calibration Theory}

Vector network analyzer calibration relies on error models that represent the systematic errors in the measurement system. For one-port measurements (S11), a three-term error model is typically used:

\begin{equation}
    S_{11,measured} = \frac{e_{00} + \frac{e_{01}e_{10}S_{11,actual}}{1-e_{11}S_{11,actual}}}{1}
\end{equation}

where:
\begin{itemize}
    \item $e_{00}$ is the directivity error (signal leakage directly from source to receiver)
    \item $e_{11}$ is the source match error (reflections due to impedance mismatch)
    \item $e_{01}e_{10}$ is the reflection tracking error (frequency response variation)
\end{itemize}

By measuring known standards (Open, Short, Load), we can determine these error terms and subsequently correct the measurements of unknown devices.

\subsection{Calibration Steps Using PocketVNA}

We performed the following calibration steps using the PocketVNA software:

\begin{enumerate}
    \item Selected the "Calibration" option from the toolbar in the PocketVNA software
    \item Set the frequency range to match our experiment parameters (1600-2500 MHz, 1001 steps)
    \item Selected "Port 1 Reflection" calibration type for our S11 measurements
    \item Connected the Open standard to port 1 and performed the scan
    \item Connected the Short standard to port 1 and performed the scan
    \item Connected the Load standard to port 1 and performed the scan
    \item Saved the calibration file and applied it to subsequent measurements
\end{enumerate}

For two-port measurements (S21), we additionally performed a Through calibration by connecting a cable or Through standard between ports 1 and 2.

\subsection{Calibration Verification}

After calibration, we verified its effectiveness by:

\begin{itemize}
    \item Measuring the calibration standards again and plotting them on a Smith chart
    \item Confirming that the Open standard appears at the rightmost point of the Smith chart (infinite impedance)
    \item Confirming that the Short standard appears at the leftmost point of the Smith chart (zero impedance)
    \item Confirming that the Load standard appears at the center of the Smith chart (characteristic impedance, typically 50Ω)
    \item For two-port calibration, verifying that the Through standard gives approximately 0 dB insertion loss across the frequency band
\end{itemize}

\section{Signal Processing Workflow}

\subsection{Data Acquisition}

Data acquisition was performed using the PocketVNA software, which provides both frequency-domain and time-domain representations of the measured signals. The following steps were followed:

\begin{enumerate}
    \item Set up the frequency sweep parameters (1600-2500 MHz, 1001 points)
    \item Applied calibration to ensure accurate measurements
    \item Collected complex S-parameter data (magnitude and phase) at each frequency point
    \item Exported the data for further processing in MATLAB
\end{enumerate}

\subsection{Signal Processing Chain}

The signal processing workflow consisted of the following steps:

\begin{enumerate}
    \item \textbf{Frequency-Domain Windowing}: Applied a Kaiser window with $\beta = 6$ to reduce sidelobes in the range profile.
    \item \textbf{Zero-Padding}: Extended the frequency-domain data with zeros to increase the number of points to 4096 for finer range resolution in the IFFT output.
    \item \textbf{IFFT Transformation}: Applied the Inverse Fast Fourier Transform to convert from frequency domain to time/range domain.
    \item \textbf{Range Calculation}: Converted the time axis to distance using the propagation velocity in the medium.
    \item \textbf{Background Subtraction}: Subtracted the response of an empty scene to enhance target detection.
    \item \textbf{Range Profile Analysis}: Identified peaks in the range profile corresponding to targets.
\end{enumerate}

\subsection{Target Detection Algorithm}

For target detection, we implemented a constant false alarm rate (CFAR) detector. The CA-CFAR (Cell Averaging CFAR) algorithm computes the detection threshold as:

\begin{equation}
T_{\text{CFAR}}(m, n) = \alpha \cdot \frac{1}{N_{\text{ref}}} \sum_{(i,j) \in \mathcal{R}} |S(m+i, n+j)|^2
\end{equation}

where $S(m,n)$ represents the complex signal at range bin $m$ and cross-range bin $n$, $\mathcal{R}$ denotes the reference cells surrounding the cell under test (excluding guard cells), $N_{\text{ref}}$ is the number of reference cells, and $\alpha$ is a scaling factor determined by the desired false alarm probability:

\begin{equation}
\alpha = N_{\text{ref}} \cdot (P_{fa}^{-1/N_{\text{ref}}} - 1)
\end{equation}

\section{Experimental Results}

\subsection{Range Resolution Verification}

To verify the system's range resolution, two identical targets were placed at varying distances from each other, and the minimum distinguishable separation was determined. The theoretical range resolution of 16.7 cm was confirmed experimentally, with two targets being clearly distinguishable when separated by approximately 18 cm.

\subsection{Target Detection Performance}

We evaluated the target detection performance using various metrics:

\begin{itemize}
    \item \textbf{Position Error}: Root Mean Square Error (RMSE) between estimated and true positions:
    \begin{equation}
    \text{RMSE} = \sqrt{\frac{1}{N} \sum_{i=1}^{N} \|\hat{\mathbf{x}}_i - \mathbf{x}_i\|^2}
    \end{equation}
    
    \item \textbf{Probability of Detection}: Fraction of actual targets detected within tolerance radius $r_{\text{tol}}$:
    \begin{equation}
    P_d = \frac{\text{Number of targets correctly detected}}{\text{Total number of targets}}
    \end{equation}
    A detection was considered correct if $\|\hat{\mathbf{x}}_i - \mathbf{x}_i\| < r_{\text{tol}}$.
    
    \item \textbf{False Alarm Rate}: Number of false detections per unit area:
    \begin{equation}
    \text{FAR} = \frac{\text{Number of false detections}}{\text{Area scanned}}
    \end{equation}
\end{itemize}

\subsection{Effect of Windowing on Range Profile}

We compared different windowing functions (Rectangle/No window, Hamming, Hanning, and Kaiser with various $\beta$ values) to assess their impact on range profile quality. The Kaiser window with $\beta = 6$ provided the best balance between sidelobe reduction and range resolution preservation.

\begin{tabular}{|l|c|c|}
\hline
\textbf{Window Function} & \textbf{Peak Sidelobe Level (dB)} & \textbf{3dB Resolution Width} \\
\hline
Rectangle (No window) & -13.3 & 1.00 (relative) \\
Hamming ($\alpha = 0.54$) & -42.7 & 1.30 \\
Hanning & -31.5 & 1.44 \\
Kaiser ($\beta = 3$) & -25.7 & 1.02 \\
Kaiser ($\beta = 6$) & -44.1 & 1.22 \\
Kaiser ($\beta = 13$) & -90.0 & 1.75 \\
\hline
\end{tabular}

\subsection{Detection Range and Accuracy}

The system was able to reliably detect targets at ranges from 0.5 meters to approximately 30-50 meters in free space. In soil with an estimated relative permittivity of 4, the maximum detection depth was reduced to approximately 15-25 meters due to signal attenuation.

\section{Limitations of the Current Design}

The current SFCW radar system faces several limitations:

\subsection{Environmental and Material Constraints}

The system's performance degrades substantially in high-moisture soils due to increased signal attenuation. Soil with moisture content exceeding 20\% by volume can reduce the effective penetration depth by over 60\%, particularly affecting the higher frequencies in our 1600-2500 MHz band. Additionally, highly heterogeneous soils with varying mineral content or numerous rocks create complex clutter patterns that the current clutter rejection algorithms struggle to fully mitigate.

The selected frequency range (1600-2500 MHz) represents a compromise between penetration depth and resolution. While this provides the $\sim$16.7 cm theoretical resolution in free space, it limits penetration to approximately 30-50 cm in typical soils - insufficient for deeper targets. This penetration limitation is particularly problematic in clay-rich soils where attenuation can be 20-30 dB/m at these frequencies.

\section{Conclusion and Future Work}

We have successfully demonstrated a target detection system using a Vector Network Analyzer operating on SFCW principles. The system achieved a range resolution of approximately 18 cm, closely matching the theoretical value of 16.7 cm. The system successfully detected targets at various ranges with high accuracy.

Future work will focus on:

\begin{itemize}
    \item Implementing a more advanced clutter rejection algorithm to improve target detection in complex environments
    \item Extending the frequency range to improve range resolution further
    \item Developing a portable system with dedicated hardware rather than using a general-purpose VNA
    \item Implementing real-time processing capabilities
    \item Testing the system in various soil conditions to better characterize its limitations
\end{itemize}

\section{References}

\begin{enumerate}
    \item Daniels, D. J., "Ground Penetrating Radar," 2nd Ed., IET, 2004.
    \item Taylor, J. D., "Ultra-Wideband Radar Technology," CRC Press, 2001.
    \item Teppati, V., Ferrero, A., Sayed, M., "Modern RF and Microwave Measurement Techniques," Cambridge University Press, 2013.
    \item Dunsmore, J. P., "Handbook of Microwave Component Measurements: With Advanced VNA Techniques," Wiley, 2012.
    \item PocketVNA, "Software and Hardware Documentation and Manual," 2023.
\end{enumerate}

\end{document}